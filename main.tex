\documentclass{beamer}
%\usetheme{Berkeley}
% \usetheme{Berlin}
\usetheme{Boadilla}
%\usetheme{CambridgeUS}
%\usetheme{Copenhagen}
%\usetheme{Madrid}
%\usecolortheme{beaver}
\usepackage{amsmath,amssymb,amsfonts,booktabs,empheq,lmodern,mathtools,nccmath,pgfplots}
\usepackage{hyperref}
%\hypersetup{colorlinks=true,linkcolor={blue}}
\sisetup{load-configurations = abbreviations}
\pgfplotsset{width=10cm,compat=1.9}
\usepgfplotslibrary{external}
\tikzexternalize
\title{Group Theory Applications}
\subtitle{Summer Improvement Program}
\author{Vatsal}
\institute[IIT Bombay]{Department of Chemistry \\
Indian Institute of Technology Bombay}
\date{July 5 - 19, 2021}

\newtheorem{proposition}{Proposition}

\newcounter{savedenum}
\newcommand*{\saveenum}{\setcounter{savedenum}{\theenumi}}
\newcommand*{\resume}{\setcounter{enumi}{\thesavedenum}}

\definecolor{titles}{rgb}{0.4,0.2,0.6}
\definecolor{bolds}{rgb}{0.0,0.4,0.6}
\definecolor{italics}{rgb}{0.9,0.3,0.6}
\setbeamercolor{block title}{fg=titles}
\newcommand\boldtext[1]{\textcolor{bolds}{\textbf{#1}}}
\newcommand\italictext[1]{\textcolor{italics}{\textit{#1}}}

\setbeamertemplate{section in toc}{\hspace*{7em}\inserttocsection}

\AtBeginSection[]
{
  \begin{frame}{Outline}
    \tableofcontents[currentsection]
  \end{frame}
}

\begin{document}

\begin{frame}
\titlepage
\end{frame}

\section*{Outline}
\begin{frame}{Outline}
    \tableofcontents
\end{frame}

\section{Unit I - Introduction to Abstract Algebra}

\begin{frame}{Set}
    \begin{definition}
        A \boldtext{set} is a collection of objects or entities (referred to as \italictext{elements}).
    \end{definition}
    Examples: $\{\}; \{\star,\bullet,\wedge\}$, where the elements are arbitrary; $\{x \mid x \in \mathbb{Z}\cup\mathbb{Q}\}$. 
    \begin{definition}
        $B$ is a \boldtext{subset} of $A$ if all elements of $B$ also belong to $A$.
    \end{definition}
    Example: For $A=\{1,2\}$, $B (\subseteq A)$ can be $\{1,2\},\{1\},\{2\}$, or $\{\}$.
    \begin{definition}
        A \boldtext{Cartesian product} between two sets $A$ and $B$ is a set of all the ordered pairs such that one element in the pair belongs to $A$, while the other to $B$, i.e. $A\times B\coloneqq\{(a,b)\mid a\in A,$ $b\in B\}$.
    \end{definition}
    \begin{align*}
        \text{Examples: } & \text{For $A=\{1,2\},B=\{2,1\},A\times B=\{(1,2),(1,1),(2,2),(2,1)\}$;}\\ & \text{$\mathbb{R}\times\mathbb{R}=\{(x,y)\mid x,y\in\mathbb{R}\}$}.
    \end{align*}
\end{frame}

\begin{frame}{Vector Space}
    \begin{definition}
        A \boldtext{vector space} is a set $V$ of vectors, adorned with an additional feature of allowing two operations, namely, vector addition ($+$) and scalar multiplication ($\cdot$), satisfying the following vector axioms for vectors $\mathbf{u},\mathbf{v},\mathbf{w} \in {V}$ and scalars $a,b \in {F}$, (where $F=\mathbb{R}, \mathbb{C},$ etc.) :
        \begin{itemize}
            \item \italictext{Commutativity} of vector addition: $\mathbf{u}+\mathbf{v}=\mathbf{v}+\mathbf{u}$ ,
            \item \italictext{Associativity} of vector addition: $\mathbf{u}+(\mathbf{v}+\mathbf{w})=(\mathbf{u}+\mathbf{v})+\mathbf{w}$ ,
            \item \italictext{Identity} for vector addition: $\exists \mathbf{ 0}\in V \text{ s.t. } \mathbf{0}+\mathbf{v}=\mathbf{v}=\mathbf{v}+\mathbf{0} \text{,  } \forall \mathbf{ v}\in V$,
            \item \italictext{Inverse} for vector addition: $\exists \text{ }(-\mathbf{v})\in \textit{V} \text{ s.t. } (-\mathbf{v})+\mathbf{v}=\mathbf{0}=\mathbf{v}+(-\mathbf{v})$,
            \item \italictext{Closure} under both the operations: $a\cdot\mathbf{v}\in V$ and $\mathbf{v}+\mathbf{w}\in V$ ,
            \item \italictext{Distributivity} over vector addition: $a\cdot(\mathbf{v}+\mathbf{w})=a\cdot\mathbf{v}+a\cdot\mathbf{w}$ ,
            \item \italictext{Distributivity} over scalar addition: $(a+b)\cdot\mathbf{v}=a\cdot\mathbf{v}+b\cdot\mathbf{v}$ ,
            \item \italictext{Associativity} of scalar multiplication: $(ab)\cdot\mathbf{v}=a\cdot(b\cdot\mathbf{v})$ ,
            \item \italictext{Identity} for scalar multiplication: $\exists \text{ } 1 \in F \text{ such that } 1\cdot\mathbf{v}=\mathbf{v}$ .
        \end{itemize}
    \end{definition}
\end{frame}

\begin{frame}{Examples of a vector space}
    \begin{itemize}
        \item $\mathbb{R}\coloneqq\left\{x_1,x_2,\dots,x_n,\dots\right\}$ ;
        \item $\mathbb{C}$ ;
        \item $\mathbb{R}^n\coloneqq\left\{\mathbf{x}\mid\mathbf{x} = \begin{bmatrix} x_1 \\ x_2 \\ \vdots \\ x_n \end{bmatrix}\right\}$ , or ,  $\left\{\mathbf{x}\mid\mathbf{x} =\begin{bmatrix} x_1 & x_2 & \cdots & x_n \end{bmatrix}\right\}$ , etc. ;
        \item $\mathbb{R}^{m\times n}\coloneqq\left\{\mathbf{v}\mid\mathbf{v} = \begin{pmatrix}
            x_{11} & x_{12} & \cdots & x_{1n} \\
            x_{21} & x_{22} & \cdots & x_{2n} \\
            \vdots & \vdots & \ddots & \vdots \\
            x_{m1} & x_{m2} & \cdots & x_{mn}
        \end{pmatrix}\right\}$ ;
        \item \italictext{Function space}: The set of functions $\{f\mid f:V\xrightarrow{}W\}$ , where ${V}$ and ${W}$ are vector spaces (as defined) and the operations (vector addition and scalar multiplication) are defined pointwise ;
        \item \italictext{Dual space}, $V^*\coloneqq\{g\mid g:V\xrightarrow{}\textit{F }\}$ , \textit{g} is called a \italictext{functional} ;
        \item $C^n[a,b]$ : The set of all real-valued $n$ times continuously differentiable functions defined on the interval $[a,b]$ .
    \end{itemize}
\end{frame}

\begin{frame}{Linear Subspace}
    \begin{definition}
        A \boldtext{linear subspace} $W$ is a non-empty subset of vector space $V$, adorned with the same operations, vector addition ($+$) and scalar multiplication ($\cdot$), which satisfy the same vector axioms, i.e. in addition to the axioms for $V$, for vectors $\mathbf{u},\mathbf{v} \in {W}$ $(\subseteq V)$ and scalars $a,b \in {F}$, (where $F=\mathbb{R}, \mathbb{C},$ etc.) :
        \begin{itemize}
            \item $a\cdot\mathbf{u}+b\cdot\mathbf{v}\in W$ .
        \end{itemize}
    \end{definition}
    Examples:
    \begin{itemize}
        \item Set of all vectors with second component $0$, forms a subspace of $\mathbb{R}^3$,
        \item Set of all real symmetric $n\times n$ matrices forms a subspace of $\mathbb{R}^{n\times n}$,
        \item $C^r[a,b]$ is a subspace of $C^s[a,b]$ for all $r>s$,
        \item \italictext{Null space (or kernel)} of a matrix $A_{m\times n}$ : Solution space $\mathbf{x}\in\mathbb{R}^n$ of the homogenous system $A\mathbf{x}=\mathbf{0}$ is a subspace of $\mathbb{R}^n$.
    \end{itemize}
\end{frame}

\begin{frame}{Inner Product Space}
    \begin{definition}
        A vector space $V$ is called an \boldtext{inner product space} if it admits a map $\langle\cdot,\cdot\rangle:V\times V\to F$, called an \italictext{inner product}, satisfying the following axioms for vectors $\mathbf{u},\mathbf{v},\mathbf{w} \in V$ and scalars $a,b \in F$, (where $F=\mathbb{R}, \mathbb{C},$ etc.) :
        \begin{itemize}
            \item \italictext{Linearity} in the first argument: $\langle a\cdot\mathbf{u}+b\cdot\mathbf{v},\mathbf{w}\rangle=a\langle \mathbf{u},\mathbf{w}\rangle+b\langle\mathbf{v},\mathbf{w}\rangle$ ,
            \item \italictext{Conjugate symmetry} or \italictext{Hermitian symmetry}: $\langle\mathbf{u},\mathbf{v}\rangle = \overline{\langle \mathbf{v},\mathbf{u}\rangle}$ ,
            \item \italictext{Positive definiteness}: $\langle\mathbf{v},\mathbf{v}\rangle>0$, if $\mathbf{v}\neq\mathbf{0}$ .
        \end{itemize}
    \end{definition}
    Alternatively, an inner product can be written with linearity in the second argument and conjugate linearity (antilinearity) in the first argument, as is the case with Dirac's bra-ket notation, where $\langle\mathbf{v},\mathbf{w}\rangle=\braket{\mathbf{w}}{\mathbf{v}}$.
\end{frame}

\begin{frame}{Direct Sum of matrices}
    \begin{definition}
        \boldtext{Direct sum} $\oplus$ of matrices $\mathbf{A}_{m\times n}$ and $\mathbf{B}_{p\times q}$ is an ($m+p$)$\times$($n+q$) matrix, defined as, $\mathbf{A}_{m\times n}\oplus\mathbf{B}_{p\times q}\coloneqq
        \begin{pmatrix}
            \mathbf{A}_{m\times n} & \mathbf{O}_{m\times q} \\
            \mathbf{O}_{p\times n} & \mathbf{B}_{p\times q}
        \end{pmatrix}$, where $\mathbf{O}$ is a null matrix.
    \end{definition}
    The direct sum of square matrices is thus, a square block diagonal  matrix.\\
    Elements in the direct sum of vector spaces $V$ and $W$ of matrices can be represented as direct sum of matrices $\mathbf{v}\oplus\mathbf{w}$, such that $\mathbf{v}\in V$, $\mathbf{w}\in W$.
\end{frame}

\begin{frame}{Direct Product of matrices}
    \begin{definition}
        \boldtext{Direct product} $\otimes$ of matrices $\mathbf{A}_{m\times n}$ and $\mathbf{B}_{p\times q}$ (or \italictext{Kronecker product}) is an ($mp$)$\times$($nq$) matrix, defined as, $\mathbf{A}_{m\times n}\otimes\mathbf{B}_{p\times q}\coloneqq
        \begin{pmatrix}
            a_{11}\mathbf{B} & \cdots & a_{1n}\mathbf{B} \\
            \vdots & \ddots & \vdots \\
            a_{m1}\mathbf{B} & \cdots & a_{mn}\mathbf{B}
        \end{pmatrix}$.
    \end{definition}
    The Kronecker product $V\otimes W$ between vector spaces $V$ and $W$ of matrices is a type of tensor product (called \italictext{tensor direct product}). It is a vector space of tensors $\mathbf{v}\otimes\mathbf{w}$, such that $\mathbf{v}\in V$, $\mathbf{w}\in W$.\\
    Example: Let orbital angular momentum operator $\mathbf{L}$ and spin angular momentum operator $\mathbf{S}$ belong to state spaces $\mathcal{H}_1$ and $\mathcal{H}_2$, respectively. The total angular momentum operator $\mathbf{J}$ then belongs to the direct product space $\mathcal{H}=\mathcal{H}_1\otimes\mathcal{H}_2$ and is given by $\mathbf{J}=\mathbf{L}\otimes\mathbf{I}_2+\mathbf{I}_1\otimes\mathbf{S}$, where $\mathbf{I}_1$ and $\mathbf{I}_2$ are identity operators in $\mathcal{H}_1$ and $\mathcal{H}_2$, respectively. For repeated use, an abridged notation of $\mathbf{J}=\mathbf{L}+\mathbf{S}$ is adopted once it is understood as to which operator resides in which space.
\end{frame}

\begin{frame}{Group}
    \begin{definition}
        A \boldtext{group}, denoted as $(G,\star )$, is a set $G$ of elements, which admits a binary operation $\star$ (called the \italictext{group operation}), satisfying the following group axioms for elements $a, b, c \in G$ :
        \begin{itemize}
            \item \italictext{Closure}: $a\star b\in G$ ,
            \item \italictext{Associativity}: $a\star(b\star c)=(a\star b)\star c$ ,
            \item \italictext{Identity}: $\exists \text{ } e\in G \text{ such that } e\star a=a=a\star e \text{,  } \forall \text{ } a\in G$ ,
            \item \italictext{Inverse}: $\exists \text{ } a^{-1}\in G \text{ such that } a^{-1}\star a=e=a\star a^{-1}$ .
        \end{itemize}
    \end{definition}
    For multiplicative groups, the group operation (multiplication, $\cdot$) is implied by mere juxtaposition of elements, i.e. $a\cdot b\equiv ab$.
\end{frame}

\begin{frame}{Examples of a group}
    \begin{itemize}
        \item $(\mathbb{Z},+)$: set of integers under addition,
        \item $(\mathbb{R},+)$: set of real numbers under addition,
        \item $(\mathbb{Q},+)$: set of rational numbers under addition,
        \item $(\mathbb{R}\backslash \{0\},\times)$: set of non-zero real numbers under multiplication,
        \item $(\mathbb{Q}\backslash \{0\},\times)$: set of non-zero rational numbers under multiplication,
        \item $(\mathbb{R}^+,\times)$: set of positive real numbers under multiplication,
        \item $(\mathbb{Q}^+,\times)$: set of positive rational numbers under multiplication,
        \item $(V,+)$, where set $V$ forms a vector space and $+$ is vector addition, 
        \item $(k\mathbb{Z,+})$: set of integer multiples of $k\in\mathbb{Z}$ under addition,
        \item $\mathbf{GL}(n,F)$: set of $n\times n$ invertible matrices under matrix multiplication,
        \item $\mathbf{GL}(V)$: set of all bijective linear transformations $V\xrightarrow{}V$ under function composition $\circ$,
        \item $\mathbf{U}(n)$: set of $n\times n$ unitary matrices under matrix multiplication,
        \item $\mathbf{U}(V)$: set of all unitary transformations $V\xrightarrow{}V$ (i.e. inner product preserving bijections) under function composition $\circ$ .
    \end{itemize}
\end{frame}

\begin{frame}{Subgroup}
    \begin{definition}
        $(H,\star)$ is called a \boldtext{subgroup} if $H$ is a subset of set $G$ that forms a group $(G,\star )$, such that $H$ too forms a group under the same group operation $\star$.
    \end{definition}
    A subgroup-group relation is denoted by the symbol $\leq$ .\\
    Examples:
    \begin{itemize}
        \item $(\mathbb{Z},+)$ is a subgroup of $(\mathbb{Q},+)\leq(\mathbb{R},+)\leq(\mathbb{C},+)$,
        \item $(\mathbb{Q}^+,\times)$ is a subgroup of $(\mathbb{R}^+,\times)$,
        \item A linear subspace of a vector space $V$ is a subgroup of $(V,+)$,
        \item $\mathbf{SL}(n,F)$: subgroup of $\mathbf{GL}(n,F)$, of matrices with unit determinant,
        \item $\mathbf{U}(n)$: subgroup of $\mathbf{GL}(n,\mathbb{C})$,
        \item $\mathbf{SU}(n)$: subgroup of $\mathbf{U}(n)$ and $\mathbf{SL}(n,\mathbb{C})$, of matrices with $\mathbf{det}=1$,
        \item $\mathbf{O}(n)$: set of $n\times n$ orthogonal matrices forms a subgroup of $\mathbf{GL}(n,\mathbb{R})$,
        \item $\mathbf{SO}(n)$: subgroup of $\mathbf{O}(n)$ and $\mathbf{SL}(n,\mathbb{R})$, of matrices with $\mathbf{det}=1$.
    \end{itemize}
\end{frame}

\begin{frame}{Direct Product of groups}
    \begin{definition}
        \boldtext{Direct product} of groups $(G,\star)$ and $(H,\bullet)$ is the group $(G\times H,\ast)$, where the underlying set is given by the Cartesian product $G\times H$ and the binary operation on the direct product group is defined component-wise:
        \begin{itemize}
            \item $(g_1,h_1)\ast(g_2,h_2)=(g_1\star g_2,h_1\bullet h_2)$, $\forall$ $g_1,g_2\in G$, $h_1,h_2\in H$.
        \end{itemize}
    \end{definition}
    \begin{description}
        \item[Exercise] Verify that the operation $\ast$ on the set $G\times H$ satisfies the group axioms. 
    \end{description}
    Example: $(\mathbb{R}\times\mathbb{R},+)$, set of 2-component vectors under vector addition.
\end{frame}

\begin{frame}{Exercise 1}
    \begin{enumerate}
        \item Argue in favour or opposition:
        \begin{itemize}
            \item Set containing just the identity element for vector addition, along with that operation and scalar multiplication, forms a vector space. \textbf{(T)}
            \item Set containing no element, along with vector addition and scalar multiplication, forms a vector space. \textbf{(F)}
            \item Set of integers does not form a group under multiplication. \textbf{(T)}
            \item Set of real (or rational) numbers does not form a group under multiplication. \textbf{(T)}
            \item Set of positive real (or rational) numbers does not form a group under addition. \textbf{(T)}
            \item Set of non-zero real (or rational) numbers does not form a group under addition. \textbf{(T)}
            \item Set of complex numbers does not form a group under addition. \textbf{(F)} 
            \item Set of non-zero complex numbers does not form a group under multiplication. \textbf{(F)}
            \item Set of non-zero real numbers forms a group under division. \textbf{(F)}
            \item Set containing just the identity element for a binary operation forms a group under that operation. \textbf{(T)}
            \item Set containing no element forms a group under any operation. \textbf{(F)}
        \end{itemize}
        \saveenum
    \end{enumerate}
\end{frame}

\begin{frame}{Exercise 1 (contd.)}
    \begin{enumerate}
        \resume
        \item Prove the following:
        \begin{itemize}
            \item Any line passing through the origin of the plane $\mathbb{R}^2$ is its subspace.
            \item $\mathbb{R}$ is not a subspace of $\mathbb{C}$.
            \item The inner product $\langle\mathbf{v},\mathbf{v}\rangle$ is a real number for all $\mathbf{v}$.
            \item An inner product satisfies conjugate linearity in the second argument.
            \item Operators residing in different spaces commute.
            \item Not every element of an arbitrary tensor product of vector spaces, $V\otimes W$, may be expressible as a tensor product of vectors, $\mathbf{v}\otimes\mathbf{w}$, such that vectors $\mathbf{v}\in V$, $\mathbf{w}\in W$.
            \item The identity element of the group is unique.
            \item The inverse element for each element of the group is unique.
            \item An intersection of subgroups of a group is also its subgroup.
            \item A union of subgroups of a group is not necessarily its subgroup.
        \end{itemize}
    \end{enumerate}
\end{frame}

\section{Unit II - Key Elements of Group Theory}

\begin{frame}{Linear Span and Basis}
    \begin{definition}
        Given a vector space $V$ and a set of vectors $S\coloneqq\left\{\mathbf{v}_1,\dots,\mathbf{v}_n\right\}$, the \boldtext{linear span} of set $S$, denoted $\mathbf{span}(S)$, is the set of all finite linear combinations of the elements of $S$, i.e. for vectors $\mathbf{v}_i \in {V}$ and scalars $a_i \in {F}$ (where $F=$ $\mathbb{R}, \mathbb{C},$ etc.), $\mathbf{span}(S)=\left\{\sum\limits_{i=1}^{k}a_i\cdot\mathbf{v}_i\mid\mathbf{v}_i \in {S},\text{ }k\in\mathbb{N}\right\}$.
    \end{definition}
    \vspace{-0.05em}
    The linear span of a set of vectors is therefore a vector space (particularly, a linear subspace $W$ of $V$) and $S\subseteq W$.
    \vspace{-0.05em}
    \begin{lemma}
        $\mathbf{span}(S)$ is the smallest subspace of $V$ that contains the spanning set $S$.
    \end{lemma}
    \vspace{-0.05em}
    $V$ is said to have been spanned by $S$ and $S$ is called a \italictext{generating set} of $V$.
    \begin{definition}
        Smallest possible generating set (or a minimal spanning set) $\{\mathbf{e}_i\}$ is called a \boldtext{basis}, i.e. set $\{\mathbf{e}_i\}$ is a maximal linearly independent set of \italictext{basis vectors}.
    \end{definition}
\end{frame}

\begin{frame}{Generating Set of a group and Cyclic Group}
    \begin{definition}
        A \boldtext{generating set of a group} is a subset $S$ of group $G$ such that each element of $G$ can be expressed as some combination (under the group operation $\star$) of finitely many elements of $S$ and their inverses.
    \end{definition}
    $G$ is said to be \italictext{generated} by $S$ and the elements of $S$ are called \italictext{generators}.\\
    To signify the same, $G$ is denoted as $\langle S\rangle$.\\
    By convention, $\langle\varnothing\rangle$ is taken to be the trivial group $(\{e\},\star)$.\\
    For a finite generating set $S$, $G$ is said to have been \italictext{finitely generated}.
    \begin{definition}
        A \boldtext{cyclic group} $C$ is a group generated by a single element $g$. Therefore, $\forall$ $k\in\mathbb{Z}$,  $\exists$ $c\in C$ such that $c=g^k$.
    \end{definition}
\end{frame}

\begin{frame}{Normal Subgroup}
    \begin{definition}
        Two elements $a,b\in G$ are \boldtext{similar} (or \italictext{conjugate} to each other) if there exists $g\in G$, such that $a=gbg^{-1}$ and the relation is called a \boldtext{similarity transformation} (or \italictext{conjugacy}).
    \end{definition}
    \begin{definition}
        A subgroup $gHg^{-1}\leq G$ is called a \boldtext{conjugate subgroup} of the subgroup $H\leq G$, defined for a fixed $g\in G$ as $gHg^{-1}\coloneqq\{ghg^{-1} \mid h \in H\}$.
    \end{definition}
    \begin{definition}
        A subgroup $N\leq G$ is called a \boldtext{normal subgroup} (also called \italictext{invariant or self-conjugate subgroup}) of $G$ iff   $\forall$ $g\in G$, $gNg^{-1}=N$ and is denoted as $N\vartriangleleft G$.
    \end{definition}
    \begin{definition}
        A \boldtext{simple group} $G$ is a group whose only normal subgroups are $\{e\}$ and $G$.  
    \end{definition}
\end{frame}

\begin{frame}{Conjugacy Class and Center of a group}
    \begin{definition}
        In a group $G$, the \boldtext{conjugacy class} of an element $x\in G$ is defined as $\text{cl}_G(x)\coloneqq\{gxg^{-1} \mid g \in G\}$, i.e. the set of all elements in $G$, which are conjugate to $x$. 
    \end{definition}
    \begin{definition}
        The \boldtext{center} of a group $G$ is the set $Z(G)$ of all the elements that commute with every element of $G$, i.e. $Z(G)=\{z\in G \mid zgz^{-1}=g$  $\forall$ $g\in G\}$.
    \end{definition}
    Equivalently, the center is the set of elements, which are the only elements in their respective conjugacy classes, i.e. $x\in Z(G)\iff\text{cl}_G(x)=\{x\}$.
    \begin{theorem}
        The center of a group $G$ is its normal subgroup, $Z(G)\vartriangleleft G$.
    \end{theorem}
\end{frame}

\begin{frame}{Abelian Group}
    \begin{definition}
        An \boldtext{abelian group} $(G,\star )$ is a group in which the elements of set $G$ commute under the binary operation $\star$ , i.e. in addition to the group axioms, for elements $a, b \in G$, following is satisfied :
        \begin{itemize}
            \item \italictext{Commutativity}: $a\star b=b\star a$ .
        \end{itemize}
    \end{definition}
    For an abelian group $G$, $Z(G) = G$.
    \begin{definition}
        \boldtext{Direct sum} of abelian groups $(G,\star)$ and $(H,\bullet)$ is the same as their direct product, i.e. $G\oplus H=(G\times H,\ast)$, such that $\ast$ is defined component-wise:
        \begin{itemize}
            \item $(g_1,h_1)\ast(g_2,h_2)=(g_1\star g_2,h_1\bullet h_2)$, $\forall$ $g_1,g_2\in G$, $h_1,h_2\in H$.
        \end{itemize}
    \end{definition}
    \begin{description}
        \item[Exercise] Verify that the operation $\ast$ on the set $G\times H$ is commutative. 
    \end{description}
\end{frame}

\begin{frame}{Order}
    \begin{definition}
        \boldtext{Order of an element} $g\in G$ is the smallest positive integer $m$ such that $g^m\equiv g\star g\star g\star\dots\text{($m$ times)}=e$, where $e$ is the identity element of $G$. 
    \end{definition}
    If no such $m$ exists the order of the element is said to be infinite.
    \begin{definition}
        \boldtext{Order of a group} $G$, denoted by $\lvert G\rvert$, is the number of elements in its set.
    \end{definition}
    If $\lvert G\rvert$ is finite, $G$ is called a \italictext{finite group}.
    \begin{definition}
        \boldtext{Conjugacy class order} is the number of elements in a conjugacy class.
    \end{definition}
    \begin{definition}
        \boldtext{Class number} of a group is the number of distinct conjugacy classes.
    \end{definition}
\end{frame}

\begin{frame}{Lagrange's Theorem}
    \begin{theorem}
        The order of every subgroup of a finite group $G$ is a factor of $\lvert G\rvert$.
    \end{theorem}
    \begin{corollary}
        The order of every element of a finite group $G$ is a factor of $\lvert G\rvert$.
    \end{corollary}
    \begin{corollary}
        $\forall$ $g\in G$, $g^{\lvert G\rvert}=e$.
    \end{corollary}
    \begin{corollary}
        Every group $G$ of prime order $(\text{i.e. }\lvert G\rvert\text{ being prime})$ is cyclic, hence simple.
    \end{corollary}
    \begin{corollary}
        The order of every conjugacy class of a finite group $G$ is a factor of $\lvert G\rvert$.
    \end{corollary}
\end{frame}

\begin{frame}{Sylow Theorems}
    \begin{theorem}
        For a prime number $p$ and a finite group $G$, if $\lvert G\rvert=p^nq$ $(p$ doesn't divide $q)$, there exists a subgroup of $G$, of order $p^k$ for each $k\in\{1,\dots,n\}$.
    \end{theorem}
    Subgroup of order $p^n$ is known as a Sylow $p$-subgroup of $G$.\\
    There exists a subgroup of $G$, of order $q$, an example of a Hall subgroup.\\
    If $q=1$, $G$ is called a $p$-group and $\forall g\in G$, order $\lvert g\rvert$ is some power of $p$.
    \begin{corollary}[Cauchy's theorem]
        For a finite group $G$, if $p$ is a prime factor of $\lvert G\rvert$, there exists an element (and thus a cyclic subgroup generated by this element) of order $p$ in $G$.
    \end{corollary}
    \begin{theorem}
        For a prime number $p$, all Sylow $p$-subgroups are conjugate to each other.
    \end{theorem}
    \begin{theorem}
        Number $n_p$ of Sylow $p$-subgroups of $G$ divides $\lvert G\rvert$, while $p$ divides $(n_p-1)$.
    \end{theorem}
\end{frame}

\begin{frame}{Exercise 2}
    \begin{enumerate}
        \item Prove the following:
        \begin{itemize}
            \item A cyclic group is an abelian group of the same order as its generator.
            \item A group with exactly 2 subgroups is a cyclic group of prime order.
            \item A simple abelian group is a cyclic group of prime order.
            \item A non-trivial simple cyclic group of prime order has exactly 2 subgroups.
            \item In any group, the identity element is the only element in its class.
            \item In an abelian group, each element is the only element in its class.
            \item Every subgroup of an abelian group is a normal subgroup.
            \item Every cyclic group of prime order is a simple group.
            \item Elements that belong to the same conjugacy class have the same order.
            \item Given $\text{cl}_G(a)=\text{cl}_G(b)$, it holds that $\text{cl}_G(a^m)=\text{cl}_G(b^m)$, where $m\in\mathbb{Z}$.
            \item For $a,b\in G$, the order of elements $\lvert ab^{-1}\rvert=\lvert b^{-1}a\rvert=\lvert a^{-1}b\rvert=\lvert ba^{-1}\rvert$.
            \item For $a,b\in G$, if order of element $a$ (or $b$) is $2$, $\lvert ab^{-1}\rvert=\lvert ab\rvert=\lvert ba\rvert$.
        \end{itemize}
        \item Argue in favour or opposition:
        \begin{itemize}
            \item Not all elements are conjugate to themselves. \textbf{(F)}
            \item Every group is a normal subgroup of itself. \textbf{(T)}
            \item Every group has the trivial group as a normal subgroup. \textbf{(T)}
            \item Conjugacy classes of a group can be overlapping sets. \textbf{(F)}
        \end{itemize}
        \saveenum
    \end{enumerate}
\end{frame}

\begin{frame}{Exercise 2 (contd.)}
    \begin{enumerate}
        \resume
        \item Argue in favour or opposition:
        \begin{itemize}
            \item Any group can be partitioned into its classes. \textbf{(T)}
            \item Any subgroup of $G$ can be partitioned into the classes of $G$. \textbf{(F)}
            \item A normal subgroup of $G$ can be partitioned into the classes of $G$. \textbf{(T)}
            \item If $H\leq G$, $N\vartriangleleft G$ and $N\leq H$, then $N\vartriangleleft H$. \textbf{(T)}
            \item The center of a group is not necessarily abelian. \textbf{(F)}
            \item The center of a non-abelian simple group is trivial. \textbf{(T)}
            \item For a prime number $p$ and a finite group $G$, if $\lvert G\rvert=p^nq$ $(p$ doesn't divide $q)$, there exists an element (and thus a cyclic subgroup generated by this element) in $G$, of order $p^k$ for each $k\in\{1,\dots,n\}$. \textbf{(T)}
            \item For a prime number $p$, a subgroup $H$ of $G$, of order $\lvert H\rvert=p^k$ is a subgroup of some Sylow $p$-subgroup of $G$. \textbf{(T)}
            \item For a prime number $p$, a normal subgroup $N$ of $G$, of order $\lvert N\rvert=p^k$ is a subgroup of every Sylow $p$-subgroup of $G$. \textbf{(T)}
            \item If $n_p=1$, the Sylow $p$-subgroup of a group is a normal subgroup. \textbf{(T)}
        \end{itemize}
    \end{enumerate}
\end{frame}

\section{Unit III - Inheritance of Group Structure}

\begin{frame}{Cayley Table}
    A Cayley table is a succinct representation of the group structure.
    \begin{definition}
        A \boldtext{Cayley table} is a square table used to describe the structure of a finite group by arranging the group elements following the group operation $\star$.
    \end{definition}
    Convention: An element (say $a$) that labels a row appears before an element (say $b$) that labels a column, in the corresponding entry of the table, i.e. as the product $a\star b$ (instead of an alternate convention of $b\star a$).
    \begin{theorem}
        Each row and each column in a Cayley table lists every group element exactly once. This is known as the rearrangement theorem.
    \end{theorem}
    Caution: A Cayley table doesn't admit associativity as an axiom for its construction. Therefore, any structure (formed by the elements of a set under the defined operations) obtained through the application of the rearrangement theorem must still be explicitly checked for associativity before assuming it to be a group structure.
\end{frame}

\begin{frame}{Examples of a Cayley table}
    \centering
    \begin{table}[]
        \centering
        \setlength\extrarowheight{3pt}
        \begin{tabular}{|c|c|c|c|}
            \hline\textbf{G} & \textbf{\textit{a}} & \textbf{\textit{b}} & \textbf{\textit{c}} \\ \hline
            \textbf{\textit{a}} & $a^2$ & $ab$ & $ac$ \\ \hline
            \textbf{\textit{b}} & $ba$ & $b^2$ & $bc$ \\ \hline
            \textbf{\textit{c}} & $ca$ & $cb$ & $c^2$ \\ \hline
        \end{tabular}
        \caption{Multiplication convention}
        %\label{tab:my_label}
    \end{table}
    \begin{columns}
        \column{0.5\textwidth}
        \centering
        $\text{C\textsubscript{4}}=\langle\{a\}\rangle$, $a^4=e$
        \begin{table}[]
            \centering
            \setlength\extrarowheight{3pt}
            \begin{tabular}{|c|c|c|c|c|}
                 \hline\textbf{C\textsubscript{4}} & \textbf{\textit{e}} & \textbf{\textit{a}} & \textbf{\textit{a\textsuperscript{2}}} & \textbf{\textit{a\textsuperscript{3}}} \\ \hline
                 \textbf{\textit{e}} & $e$ & $a$ & $a^2$ & $a^3$ \\ \hline
                 \textbf{\textit{a}} & $a$ & $a^2$ & $a^3$ & $e$\\ \hline
                 \textbf{\textit{a\textsuperscript{2}}} & $a^2$ & $a^3$ & $e$ & $a$ \\ \hline
                 \textbf{\textit{a\textsuperscript{3}}} & $a^3$ & $e$ & $a$ & $a^2$ \\\hline
            \end{tabular}
            \caption{C\textsubscript{4}, cyclic group of order 4}
            %\label{tab:my_label}
        \end{table}
        \column{0.5\textwidth}
        \centering
        $\text{V}=\langle\{a,b\}\rangle$, $a^2=e=b^2$, $ab=ba$
        \begin{table}[]
        \centering
        \setlength\extrarowheight{3pt}
        \begin{tabular}{|c|c|c|c|c|}
            \hline\textbf{V} & \textbf{\textit{e}} & \textbf{\textit{a}} & \textbf{\textit{b}} & \textbf{\textit{ab}} \\ \hline
            \textbf{\textit{e}} & $e$ & $a$ & $b$ & $ab$ \\ \hline
            \textbf{\textit{a}} & $a$ & $e$ & $ab$ & $b$\\ \hline
            \textbf{\textit{b}} & $b$ & $ab$ & $e$ & $a$ \\ \hline
            \textbf{\textit{ab}} & $ab$ & $b$ & $a$ & $e$ \\\hline
        \end{tabular}
        \caption{V, Klein four-group}
        %\label{tab:my_label}
    \end{table}
    \end{columns}
\end{frame}

\begin{frame}{Examples of a Cayley table (contd.)}
    \centering
    $\text{S\textsubscript{3} $\cong$ D\textsubscript{3}}=\langle\{a,b\}\rangle$, $a^3=e=b^2$, $a^{-1}b=ba$
    \begin{table}[]
        \centering
        \setlength\extrarowheight{3pt}
        \begin{tabular}{|c|c|c|c|c|c|c|}
            \hline\textbf{S\textsubscript{3} $\cong$ D\textsubscript{3}} & \textbf{\textit{e}} & \textbf{\textit{a}} & \textbf{\textit{a\textsuperscript{2}}} & \textbf{\textit{b}} & \textbf{\textit{ab}} & \textbf{\textit{a\textsuperscript{2}b}} \\ \hline
            \textbf{\textit{e}} & $e$ & $a$ & $a^2$ & $b$ & $ab$ & $a^2b$ \\ \hline
            \textbf{\textit{a}} & $a$ & $a^2$ & $e$ & $ab$ & $a^2b$ & $b$\\ \hline
            \textbf{\textit{a\textsuperscript{2}}} & $a^2$ & $e$ & $a$ & $a^2b$ & $b$ & $ab$ \\ \hline
            \textbf{\textit{b}} & $b$ & $a^2b$ & $ab$ & $e$ & $a^2$ & $a$ \\\hline
            \textbf{\textit{ab}} & $ab$ & $b$ & $a^2b$ & $a$ & $e$ & $a^2$ \\\hline
            \textbf{\textit{a\textsuperscript{2}b}} & $a^2b$ & $ab$ & $b$ & $a^2$ & $a$ & $e$ \\\hline
        \end{tabular}
        \caption{S\textsubscript{3}, symmetric group of permutations of 3 objects, isomorphic (vide infra) to \centering D\textsubscript{3}, dihedral group of symmetries of an equilateral triangle}
        %\label{tab:my_label}
    \end{table}
    \begin{description}
        \item[Exercise] Figure out the inverse and order of each element, subgroups, conjugacy classes, normal subgroups and center for each of the three examples.
    \end{description}
\end{frame}

\begin{frame}{Coset}
    \begin{definition}
        With respect to elements $g\in G$, the \boldtext{left cosets} of a subgroup $(H,\star)$ of a group $(G,\star)$ are given by $gH\coloneqq\{g\star h\mid h\in H\}$, and the \boldtext{right cosets} of $H$ are given by $Hg\coloneqq\{h\star g\mid h\in H\}$.
    \end{definition}
    Left (or right) cosets of $H\leq G$ are disjoint subsets of $G$.\\ 
    Each (left or right) coset of $H\leq G$ in $G$ has $\lvert H\rvert$ number of elements.\\
    Given an abelian group $G$, $\forall$ $H\leq G,g\in G$, $gH=Hg$.\\
    As seen already, $N\vartriangleleft G\iff gN=Ng$, $\forall$ $g\in G$.
    \begin{definition}
        The \boldtext{index} of a subgroup $H$ in a group $G$ is the number of left (or right) cosets of $H$ in $G$ and is denoted by $\lvert G:H\rvert$.
    \end{definition}
    Restating Lagrange's theorem:
    \begin{theorem}
        For a subgroup $H$ of a group $G$, $\lvert G\rvert=\lvert G:H\rvert\lvert H\rvert$.
    \end{theorem}
\end{frame}

\begin{frame}{Equivalence Class and Quotient Set}
    \begin{definition}
        The \boldtext{equivalence class} $[a]$ of an element $a$ in set $S$, with respect to an equivalence relation $\sim$, is the set of elements in $S$ that are equivalent to $a$, $[a]\coloneqq\{x\in S\mid a\sim x\}$, i.e. the binary relation $\sim$ satisfies $\forall$ $a,b,c\in S$:
        \begin{itemize}
            \item \italictext{Reflexivity}: $a\sim a$ ,
            \item \italictext{Symmetry}: $a\sim b\iff b\sim a$ ,
            \item \italictext{Transitivity}: $a\sim b\text{ and }b\sim c\implies a\sim c$ .
        \end{itemize}
    \end{definition}
    Each element of a set belongs to exactly one of its equivalence classes.
    \begin{definition}
        The set $S/{\sim}$ of all equivalence classes of a set $S$ is called a \italictext{partition} of $S$ by $\sim$ or a \boldtext{quotient set} of $S$ by $\sim$ .
    \end{definition}
    If a set $S$ has an additional structure (e.g. that of a vector space or a group) and the equivalence relation $\sim$ on $S$ is \italictext{compatible} with that structure, its quotient set $S/{\sim}$ inherits that structure.
\end{frame}

\begin{frame}{Quotient Space and Quotient Group}
    \begin{definition}
        For a vector space $V$ and its linear subspace $W$, the \boldtext{quotient space} $V/W$ is the quotient set $V/{\sim}$, where the equivalence relation $\sim$ on $V$ is defined as $\mathbf{v}_1\sim\mathbf{v}_2\iff\mathbf{v}_1-\mathbf{v}_2\in W$, i.e. $[\mathbf{v}]=\{\mathbf{v}+\mathbf{w}\mid\mathbf{w}\in W\}$. This set forms a vector space with the operations defined for vectors $\mathbf{v}_1,\mathbf{v}_2\in V$ and scalars $a\in F$, (where $F=\mathbb{R}, \mathbb{C},$ etc.), as:
        \begin{itemize}
            \item $a\cdot[\mathbf{v}_1]=[a\cdot\mathbf{v}_1]$ ,
            \item $[\mathbf{v}_1]+[\mathbf{v}_2]=[\mathbf{v}_1+\mathbf{v}_2]$ .
        \end{itemize}
    \end{definition}
    \begin{definition}
        For a group $(G,\star)$ and its normal subgroup $N$, the \boldtext{quotient group} $G/N$ is the set of all left (or right) cosets of $N$ in $G$, i.e. $G/N\coloneqq\{gN\mid g\in G\}$. This set forms a group with the group operation $\bullet$ defined $\forall g_1,g_2\in G$ as:
        \begin{itemize}
            \item $(g_1N)\bullet(g_2N)=(g_1\star g_2)N$ .
        \end{itemize}
    \end{definition}
    \begin{description}
        \item[Exercise] Verify that $\bullet$ on the set $G/N$ satisfies the group axioms.
    \end{description}
\end{frame}

\begin{frame}{Exercise 3}
    \begin{enumerate}
        \item Prove the following:
        \begin{itemize}
            \item Given a subgroup $H\leq G$, $\forall$ $g\in G$, $x\in gH\iff xH=gH$.
            \item Given a subgroup $H\leq G$, $\forall$ $z\in Z(G)$, $zH=Hz$.
            \item $\lvert\mathbb{Z}:k\mathbb{Z}\rvert=k$, for any $k\in\mathbb{N}$.
            \item Left cosets of $H\leq G$ form equivalence classes with respect to the equivalence relation $x\sim y\iff x^{-1}y\in H$, $\forall$ $x,y\in G$.
            \item For $K\leq H\leq G$, $\lvert G:K\rvert=\lvert G:H\rvert\lvert H:K\rvert$, without assuming Lagrange's theorem. \textbf{Hint}: Prove Lagrange's theorem.
            \item In the quotient space $V/W$, the identity for vector addition is $[\mathbf{0}]=W$.
            \item In the quotient group $G/N$, the identity is $N$ and $(gN)^{-1}=g^{-1}N$.
        \end{itemize}
        \item Argue in favour or opposition:
        \begin{itemize}
            \item A non-abelian group can have a symmetric Cayley table. \textbf{(F)}
            \item Any group in which every element is its own inverse is abelian. \textbf{(T)}
            \item A set of 6 elements, each inverse of itself, forms an abelian group. \textbf{(F)}
            \item The left coset of $H\leq G$ with respect to each $g\in G$ is the right coset of the conjugate subgroup $gHg^{-1}$ with respect to $g$. \textbf{(T)}
            \item A conjugacy class of a group is an equivalence class of the underlying set, with conjugation as its equivalence relation. \textbf{(T)}
        \end{itemize}
    \end{enumerate}
\end{frame}

\section{Unit IV - Maps involving Groups}

\begin{frame}{Structure preserving maps}
    \begin{definition}
        A \boldtext{group homomorphism} is a map $f:G\xrightarrow{}H$, i.e. from group $(G,\star)$ to $(H,\bullet)$, which preserves the group structure, viz. the identity and the group operation, by satisfying $f(g_1\star g_2)=f(g_1)\bullet f(g_2)$, $\forall g_1,g_2\in G$.
    \end{definition}
    As a result,
    \begin{itemize}
        \item $f(e_G)=e_H$, where $e_G$ and $e_H$ are identity in $G$ and $H$, respectively.
        \item $f(g_1^{-1})=(f(g_1^{}))^{-1}$, i.e. inverse in $G$ maps to inverse of image, in $H$.
    \end{itemize}
    Thus, the map $f$ is said to be \italictext{compatible} with the group structure of $G$.
    \begin{definition}
        The \boldtext{kernel} of $f:G\xrightarrow{}H$ is the set of elements in $G$ that are mapped to the identity element ($e_H$) of $H$, i.e. $\mathbf{ker}(f)\equiv\{g\mid f(g)=e_H,\text{ } g\in G\}$.
    \end{definition}
    \begin{definition}
        The \boldtext{image} of $f:G\xrightarrow{}H$ is the set of elements in $H$ to which elements of $G$ are mapped, i.e. $\mathbf{im}(f)\equiv f(G)\equiv\{f(g)\mid g\in G\}$.
    \end{definition}
\end{frame}

\begin{frame}{Structure preserving maps (contd.)}
    A group homomorphism $f:G\xrightarrow{}H$ is said to be \italictext{trivial} if $\mathbf{ker}(f)=G$.
    \begin{definition}
        A \boldtext{monomorphism} is a group homomorphism that is injective (one-one).
    \end{definition}
    \begin{definition}
        An \boldtext{epimorphism} is a group homomorphism that is surjective (onto).
    \end{definition}
    \begin{definition}
        An \boldtext{isomorphism} is a group homomorphism that is bijective (invertible).
    \end{definition}
    \begin{definition}
        An \boldtext{endomorphism} is a group homomorphism of a group to itself.
    \end{definition}
    \begin{definition}
        An \boldtext{automorphism} is a bijective (isomorphic) endomorphism of a group.
    \end{definition}
\end{frame}

\begin{frame}{Group Action}
    \begin{definition}
        A \boldtext{group action} $\phi(g,x)$ of a group $(G,\star)$ on a set $X$ is a bijective (or invertible) map $\phi:G\times X\xrightarrow{}X$, satisfying the following axioms $\forall x\in X$ :
        \begin{itemize}
            \item Identity: $\phi(e,x)=x$, where $e$ is identity in $G$,
            \item Compatibility: $\phi(g,\phi(h,x))=\phi(g\star h,x)$, $\forall$ $g,h\in G$. 
        \end{itemize}
    \end{definition}
    Set $X$, together with an action of group $G$, is called a $G$-set.\\
    The above definition is the case of a \italictext{left group action} $\phi_L:G\times X\xrightarrow{}X$, usually adopted as a convention. Alternatively, one can define a \italictext{right group action} $\phi_R:X\times G\xrightarrow{}X$, satisfying analogous axioms.
    \begin{definition}
        An \boldtext{inner automorphism} is the group action of a group $G$ on itself, given by the map $\phi_g:G\times G\xrightarrow{}G$ for a fixed $g\in G$, defined as $\phi_g(x)\coloneqq g^{-1}xg$.
    \end{definition}
\end{frame}

\begin{frame}{Orbits under group action}
    \begin{definition}
        The \boldtext{orbit} of an element $x\in X$ under action $\phi(g,x)$ of a group $(G,\star)$ is the set of elements in $X$ to which $x$ gets mapped, under the group action, i.e. $G\cdot x\coloneqq\{\phi(g,x)\mid g\in G\}$.
    \end{definition}
    \begin{definition}
        \boldtext{Length} of the orbit $G\cdot x$ of an element $x\in X$ under the action of a group $G$ is the number of elements in that orbit and is denoted as $\lvert G\cdot x\rvert$. 
    \end{definition}
    \begin{definition}
        The \boldtext{quotient of action} (denoted as $X/G$) of a group $G$ on set $X$ is the set of all orbits of elements in $X$ under the action $\phi(g,x)$ of $G$, i.e. it is the quotient set $X/{\sim}$, where the equivalence relation $\sim$ on $X$ is defined as $x\sim y\iff\exists$ $g\in G$ such that $y=\phi(g,x)$.
    \end{definition}
    Any additional structure on set $X$ is inherited by the quotient of action. 
    \begin{description}
        \item[Exercise] Verify that $\sim$ forms a partition of $X$.
    \end{description}
\end{frame}

\begin{frame}{Stabilizer}
    \begin{definition}
        For $x\in X$ and $g\in G$, $x$ is said to be a \boldtext{fixed point} of $g$ if it is invariant under the action $\phi$ of group $G$ on set $X$, i.e. $\phi(g,x)=x$.
    \end{definition}
    \begin{definition}
        The \boldtext{stabilizer} $G_x$ of a group $G$ with respect to an $x\in X$ is the set of elements in $G$ with $x$ as their fixed point, i.e. $G_x\coloneqq\{g\in G\mid\phi(g,x)=x\}$.
    \end{definition}
    \begin{definition}
        Group action is \boldtext{transitive} if there is exactly one orbit $G\cdot x=X$, $\forall$ $x\in X$.
    \end{definition}
    \begin{definition}
        Group action is \boldtext{faithful} if the intersection of the stabilizers $G_x$, $\forall$ $x\in X$, is trivial, i.e. $\bigcap\limits_{x\in X}G_x=\{e_G\}$, where $e_G$ is the identity in $G$. 
    \end{definition}
\end{frame}

\begin{frame}{Orbit-Stabilizer Theorem}
    Given a fixed $x\in X$ and action $\phi(g,x)$ of a group $(G,\star)$ on set $X$, we define a map $\alpha_x:G\xrightarrow{}X$ by $\alpha_x:g\mapsto \phi(g,x)$. Naturally, $\mathbf{im}(\alpha_x)\equiv G\cdot x$ .\\
    \begin{theorem}
        \begin{list}{\maltese}{\leftmargin=1em \itemindent=0em}
            \item For a fixed $x\in X$, there exists a bijection $f_x:G/G_x \xrightarrow{}G\cdot x$ from the set $G/G_x$ of all left cosets $gG_x$ of the stabilizer subgroup $G_x$ to the orbit $G\cdot x$ of $x$ under the action $\phi(g,x)$ of a group $(G,\star)$ on a set $X$, given by the map $f_x:gG_x\mapsto \phi(g,x)$.
            \item $\forall$ $g,h\in G$, $f_x(g)=f_x(h)$ iff $g$ and $h$ belong to the same coset of the stabilizer subgroup $G_x$, i.e. $f_x(g)=f_x(h)\iff h\in gG_x$.
            \item Length of orbit $G\cdot x$ of an element $x\in X$, under the action of a group $G$, is equal to the index of stabilizer subgroup $G_x$, of $G$ with respect to $x$, in $G$, i.e. $\lvert G\cdot x\rvert=\lvert G:G_x\rvert=\lvert G\rvert/\lvert G_x\rvert$. 
        \end{list}
    \end{theorem}
    Since stabilizers need not be normal subgroups, $G/G_x$ is not necessarily a $\text{quotient group, but simply the set of all left cosets }gG_x\text{ of the stabilizer }G_x.$
\end{frame}

\begin{frame}{Centralizer and Normalizer}
    \begin{definition}
        The \boldtext{centralizer} $C_G(S)$ of a subset $S$ in group $G$ is the set of elements in $G$ such that each element in $C_G(S)$ commutes with every element in $S$, i.e. $C_G(S)\coloneqq\{g\in G \mid gsg^{-1}=s$  $\forall$ $s\in S\}$.
    \end{definition}
    Hence, it can be equivalently said that every element in $S$ is invariant under conjugation by each element in $C_G(S)$.
    \begin{definition}
        The \boldtext{normalizer} $N_G(S)$ of a subset $S$ in group $G$ is the set of elements in $G$ such that the subset $S$ is invariant under conjugation by each element in $N_G(S)$, i.e. $N_G(S)\coloneqq\{g\in G \mid gSg^{-1}=S\}$.
    \end{definition}
    \begin{description}
        \item[Exercise] Verify that $Z(G)\leq C_G(S)\leq N_G(S)\leq G$.
    \end{description}
    Clearly, $\bigcap\limits_{S\subseteq G}C_G(S)=Z(G)$, or more evidently, $\bigcap\limits_{g\in G}C_G(\{g\})=Z(G)$.
\end{frame}

\begin{frame}{Burnside's Lemma}
    \begin{definition}
        The \boldtext{set of orbits} of a set $X$ under a group $G$ is the set of distinct orbits of all elements in $X$ under action of $G$ on $X$, i.e. $X/G\coloneqq\{G\cdot x\mid x\in X\}$.
    \end{definition}
    For each $g\in G$, let $X^g$ denote the set of fixed points of $g$ in $X$ under the action $\phi$ of group $G$ on set $X$, i.e. $X^g\coloneqq\{x\in X\mid\phi(g,x)=x\}$.
    \begin{lemma}
        The number of distinct orbits (denoted by $\lvert X/G\rvert$) of all elements in set $X$ under the action of a finite group $G$ on $X$ is given as:
        \vspace{-0.75em}
        \begin{align*}
            \lvert X/G\rvert=\frac{1}{\lvert G\rvert}\sum\limits_{g\in G}\lvert X^g\rvert \;.
        \end{align*}
    \end{lemma}
    Hence, it can be equivalently said that the number of distinct orbits of a set $X$ under the action of a finite group $G$ is equal to the average number of fixed points of an element of $G$, in $X$.
\end{frame}

\begin{frame}{Conjugacy Class Equation}
    Let a finite group $G$ have $k$ distinct conjugacy classes, out of which $l$ have conjugacy class order greater than one.
    \begin{theorem}
        Choosing a representative element $\mathcal{C}_i$ from each conjugacy class of a finite group $G$, we have
        \vspace{-0.75em}
        \begin{align*}
            \lvert G\rvert=\sum\limits_{i=1}^{k}\lvert G:C_G(\mathcal{C}_i)\rvert \;.
        \end{align*}
        Equivalently,
        \vspace{-0.75em}
        \begin{align*}
            \lvert G\rvert=\lvert Z(G)\rvert+\sum\limits_{i=1}^{l}\lvert G:C_G(\mathcal{C}_i)\rvert \;.
        \end{align*}
    \end{theorem}
    Check out \hyperlink{ex_class_eq}{\color[rgb]{0.7,0.0,0.1}this problem}, which establishes conjugacy classes as orbits under the action of a group on itself through conjugation and centralizers of the group as the stabilizer subgroups under the same group action.
\end{frame}

\begin{frame}{First Isomorphism Theorem}
    The isomorphism theorems can be proven using the \italictext{fundamental theorem on homomorphisms}, which states:
    \vspace{-0.1em}
    \begin{theorem}
        Given an arbitrary group homomorphism $f:G\xrightarrow{}H$ and an epimorphism $\phi:G\xrightarrow{}G/N$, defined by $\phi:g\mapsto gN$ for $N\vartriangleleft G$ such that $N\subseteq\mathbf{ker}(f)$, there exists a unique homomorphism $\alpha:G/N\xrightarrow{}H$ with $\alpha(gN)=f(g)$.
    \end{theorem}
    \vspace{-0.1em}
    The \italictext{first isomorphism theorem} is obtained by setting $N=\mathbf{ker}(f)$.
    \vspace{-0.1em}
    \begin{theorem}[1]
        Given a group homomorphism $f:G\xrightarrow{}H$, $\mathbf{ker}(f) \vartriangleleft G$, $\mathbf{im}(f)\leq H$ and $\mathbf{im}(f)$ is isomorphic to the quotient group $G/\mathbf{ker}(f)$, or $G/\mathbf{ker}(f)\cong\mathbf{im}(f)$.
    \end{theorem}
    \vspace{-0.1em}
    If $f$ is an epimorphism (surjective), $H$ is isomorphic to $G/\mathbf{ker}(f)$.
    \vspace{-0.1em}
    \begin{corollary}
        Given a group homomorphism $f:G\xrightarrow{}H$, the index of $\mathbf{ker}(f)$ in $G$ equals the order of $\mathbf{im}(f)$, i.e. $\lvert G:\mathbf{ker}(f)\rvert=\lvert\mathbf{im}(f)\rvert\text{.Thus, }\lvert G\rvert=\lvert\mathbf{ker}(f)\rvert\lvert\mathbf{im}(f)\rvert$.
    \end{corollary}
\end{frame}

\begin{frame}{Fundamental Theorems of abelian groups}
    \italictext{Fundamental theorem of cyclic groups}:
    \begin{theorem}
        Order of every subgroup of a finite cyclic group $G$ divides the order $\lvert G\rvert$ of group $G$ and there exists exactly one subgroup of $G$ for each divisor of $\lvert G\rvert$. 
    \end{theorem}
    \begin{lemma}
        $\text{For cyclic groups }\mathbb{Z}_k\text{ of order }k,\text{ }\mathbb{Z}_{mn}\cong\mathbb{Z}_m\oplus\mathbb{Z}_n\Leftrightarrow m \text{ and } n\text{ are coprimes.}$
    \end{lemma}
    \italictext{Fundamental theorem of finite abelian groups}:
    \begin{theorem}
        Every finite abelian group $G$ of order $\lvert G\rvert=p_1p_2\cdots p_l$ is isomorphic to a direct sum of cyclic subgroups of prime-power order, i.e. $G\cong\bigoplus\limits_{i=1}^{l} \mathbb{Z}_{p_i}$, where $\mathbb{Z}_{p_i}$ is a cyclic group of order $p_i$, a prime raised to some power.
    \end{theorem}
\end{frame}

\begin{frame}{Exercise 4}
    \begin{enumerate}
        \item Prove the following:
        \begin{itemize}
            \item Image of a group homomorphism $f:G\xrightarrow{}H$ is a subgroup of $H$.
            \item Kernel of a group homomorphism $f:G\xrightarrow{}H$ is a subgroup of $G$.
            \item Kernel of a group homomorphism is trivial iff it is a monomorphism.
            \item Kernel of a group homomorphism is a normal subgroup of the domain.
            \item Any homomorphism from a simple group must be injective or trivial.
            \item The stabilizer $G_x$ of group $G$ with respect to $x\in X$ is a subgroup of $G$.
            \item Stabilizers of elements in the same orbit are conjugate to each other.
            \item The centralizer of a set $S$ in a group $G$ is always a normal subgroup of $G$ and of the normalizer of $S$ in $G$.
            \item $C_G(C_G(S))$ contains $S$, but $C_G(S)$ need not contain $S$.
            \item For $H\leq G$, $N_G(H)$ contains $H$.
            \item\label{ex_class_eq} Under an inner automorphism on $G$, the stabilizer $G_x$ of $G$ with respect to any $x\in G$ is the centralizer $C_G(x)$, of $\{x\}$, in $G$. Also the number of distinct conjugates of $x$ in $G$ is the index $\lvert G:C_G(x)\rvert$, of $C_G(x)$, in $G$.
            \item Every finite $p$-group has a non-trivial center.
            \item Given a group epimorphism $\phi:G\xrightarrow{}G/N$, defined by $\phi:g\mapsto gN$ for $N\vartriangleleft G$, $\mathbf{ker}(\phi)=N$. Such a $\phi$ is called a \italictext{canonical homomorphism}.
        \end{itemize}
        \saveenum
    \end{enumerate}
\end{frame}

\begin{frame}{Exercise 4 (contd.)}
    \begin{enumerate}
        \resume
        \item Argue in favour or opposition:
        \begin{itemize}
            \item For a group homomorphism $f:G\xrightarrow{}H$, the preimage of a normal subgroup of $H$ is a normal subgroup of $G$. \textbf{(T)}
            \item For a group $G$, the map $f:G\xrightarrow{}G$, given by the involution $f:g\mapsto{}g^{-1}$ is always a homomorphism. \textbf{(F)}
            \item For an abelian group $G$, the map $f:G\xrightarrow{}G$, given by the involution $f:g\mapsto{}g^{-1}$ is always an automorphism. \textbf{(T)}
            \item The order of a homomorphic image of a group element doesn't necessarily divide the order of that element. \textbf{(F)}
            \item If $\mathbf{dim}(V)=n$, $\mathbf{GL}(V)$ is isomorphic to $\mathbf{GL}(n,F)$. \textbf{(T)}
            \item Given $H\leq G$, for a fixed $g\in G$, the orbit $H\cdot g$ of $g$ under the right action of $H$ on $G$ is the left coset $gH$ of $H$ in $G$, while the orbit $g\cdot H$ under the left action of $H$ on $G$ is the right coset $Hg$ of $H$ in $G$. \textbf{(T)}
            \item For an arbitrary group $G$, $C_G(G)=Z(G)$. \textbf{(F)}
            \item For singleton sets $\{g\}$, $C_G(g)\equiv C_G(\{g\})=N_G(\{g\})\equiv N_G(g)$. \textbf{(T)}
            \item For all $g\in G$, $\lvert C_G(g)\rvert$ divides the order of $g$ in $G$. \textbf{(F)}
            \item The number of distinct conjugate subgroups $gHg^{-1}$ of $H\leq G$ equals the index $\lvert G:N_G(H)\rvert$ of normalizer $N_G(H)$ of $H$ in $G$. \textbf{(T)}
            \item All abelian groups of order $27$ are isomorphic and order $21$ aren't. \textbf{(F)}
        \end{itemize}
    \end{enumerate}
\end{frame}

\section{Unit V - Representation Theory of Groups}

\begin{frame}{Matrix Transformations}
    \begin{lemma}
        Any linear transformation between finite-dimensional vector spaces can be represented by a matrix.
    \end{lemma}
    Conversely, every matrix transformation is a linear transformation.
    \begin{proposition}
        For every matrix $A_{m\times n}$ with entries in $F$ (where $F=\mathbb{R}, \mathbb{C},$ etc.), there exists the linear transformation $T:F^n\xrightarrow{}F^m$, given by $T(\mathbf{v})=A\mathbf{v}$.
    \end{proposition}
    \begin{lemma}
        Trace (determinant) of a matrix, invariant under similarity transformations, equals the sum (product) of all its eigenvalues counted with multiplicities.
    \end{lemma}
    \begin{corollary}
        Trace (determinant) of a linear transformation between finite-dimensional vector spaces is independent of the basis chosen for those vector spaces.
    \end{corollary}
\end{frame}

\begin{frame}{Diagonalizability of matrices}
    \begin{definition}
        A square matrix is called \boldtext{diagonalizable} if it is similar (associated through a similarity transformation) to a diagonal matrix.
    \end{definition}
    Equivalently, $\text{diagonalizability}\iff\text{linear independence of eigenvectors}$.\\
    The canonical approach of diagonalization is called \italictext{eigendecomposition}, in which the diagonal entries of the associated diagonal matrix are the eigenvalues of the original matrix and the columns of the diagonalizing matrix are the eigenvectors of the original matrix.
    \begin{lemma}
        If a matrix $A$ represents the linear transformation $T:V\xrightarrow{}V$ of a vector space $V$, $A$ is diagonalizable iff a basis of $V$ comprises eigenvectors of $A$.
    \end{lemma}
    \begin{proposition}
        For a finite order matrix transformation $T$ of a finite-dimensional vector space $V$, $($i.e. $T^n=\mathbb{I}$, $n\in\mathbb{N})$, a basis of $V$ comprises eigenvectors of $T$.
    \end{proposition}
\end{frame}

\begin{frame}{Spectral Theorems for matrices}
    \italictext{Spectral theorem for Hermitian matrices}:
    \begin{theorem}
        For a(n) Hermitian matrix:
        \begin{itemize}
            \item all eigenvalues are real,
            \item eigenvectors corresponding to distinct eigenvalues are orthogonal,
            \item there exists an orthogonal basis of the vector space consisting of those eigenvectors.
        \end{itemize}
    \end{theorem}
    \begin{corollary}
        Hermitian matrices are unitarily diagonalizable, i.e. for every Hermitian matrix $H$, there exists a unitary matrix $U$ such that $U^{-1}HU$ is diagonal.
    \end{corollary}
    The diagonal entries of $U^{-1}HU$ are the (real) eigenvalues of $H$ and the columns of $U$ are the (orthogonal) eigenvectors of $H$.
\end{frame}

\begin{frame}{Spectral Theorems for matrices (contd.)}
    \italictext{Spectral theorem for unitary matrices}:
    \begin{theorem}
        For a unitary matrix:
        \begin{itemize}
            \item all (complex) eigenvalues have absolute value 1,
            \item eigenvectors corresponding to distinct eigenvalues are orthogonal,
            \item there exists an orthogonal basis of the vector space consisting of those eigenvectors.
        \end{itemize}
    \end{theorem}
    \begin{corollary}
        Unitary matrices are unitarily diagonalizable, i.e. for every unitary matrix $\mathcal{U}$, there exists a unitary matrix $U$ such that $U^{-1}\mathcal{U}U$ is diagonal.
    \end{corollary}
    The diagonal entries of $U^{-1}\mathcal{U}U$ are the eigenvalues (with unit modulus) of $\mathcal{U}$ and the columns of $U$ are the (orthogonal) eigenvectors of $\mathcal{U}$.
\end{frame}

\begin{frame}{Linear Representation}
    Abstract group elements are described in terms of bijective (invertible) linear transformations (automorphisms) of vector spaces.
    \begin{definition}[1]
        A \boldtext{linear representation} $\phi$ of a group $(G,\star)$ is the group action of $G$ on a vector space $V$, i.e. $\phi:G\times V\xrightarrow{}V$, given by a map $\phi(g):v\mapsto\phi(g,v)$, satisfying the axioms for group action and linearity in the second argument.
    \end{definition}
    Dimension of vector space $V$ is called the \italictext{dimension} of the representation.
    \begin{definition}[2]
        A \boldtext{linear representation} $\rho$ of a group $(G,\star)$ is the group homomorphism $\rho:G\xrightarrow{}\mathbf{GL}(V)$ from $G$ to the general linear group $\mathbf{GL}(V)$ on a vector space $V$, given by a map $\rho(g):v\mapsto\rho(g,v)$, satisfying the axioms for group homomorphism and linearity in the second argument.
    \end{definition}
    Groups elements are usually represented by invertible matrices (sometimes called \italictext{representatives}) and the group operation by matrix multiplication.
\end{frame}

\begin{frame}{Linear Representation (contd.)}
    \begin{definition}
        \boldtext{Kernel} of a representation $\rho:G\xrightarrow{}\mathbf{GL}(V)$ is the set of preimage(s) of the identity transformation in $V$ $(\mathbf{id}_V)$, or $\mathbf{ker}(\rho)=\{g\mid \rho(g)=\mathbf{id}_V,\text{ } g\in G\}$.
    \end{definition}
    \begin{definition}
        A \boldtext{faithful representation} is a representation that is a monomorphism.
    \end{definition}
    Kernel of a faithful representation is trivial and no two elements of the group have the same representative.
    \begin{definition}
        Representations $\rho:G\xrightarrow{}\mathbf{GL}(V)$ and $\sigma:G\xrightarrow{}\mathbf{GL}(W)$ are \boldtext{isomorphic} if there exists an isomorphism $\alpha:V\xrightarrow{}W$ s.t. $\forall$ $g\in G$, $\alpha\circ\rho(g)=\sigma(g)\circ\alpha$.
    \end{definition}
    The map $\alpha$ is said to be \italictext{equivariant} when representations are isomorphic.
    \begin{definition}
        A \boldtext{unitary representation} is a group homomorphism $\rho:G\xrightarrow{}\mathbf{U}(V)$.
    \end{definition}
\end{frame}

\begin{frame}{Direct Sum of representations}
    \begin{definition}[1]
        \boldtext{Direct sum} of representations $\rho_1:G_1\xrightarrow{}\mathbf{GL}(V_1)$ and $\rho_2:G_2\xrightarrow{}\mathbf{GL}(V_2)$ is a group homomorphism $\rho_1\oplus\rho_2:G_1\times G_2\xrightarrow{}\mathbf{GL}(V_1\oplus V_2)$ such that $(\rho_1\oplus\rho_2)((g_1,g_2),v_1\oplus v_2)=\rho_1(g_1,v_1)\oplus\rho_2(g_2,v_2)$, $\forall$ $g_1 \in G_1, g_2 \in G_2$, $v_1\in V_1, v_2\in V_2$.
    \end{definition}
    \begin{definition}[2]
        \boldtext{Direct sum} of representations $(V_1,\rho_1)$ and $(V_2,\rho_2)$ of groups $G_1$ and $G_2$, respectively, is a linear representation $(V_1\oplus V_2,\rho)$ of the group $G_1\times G_2$ whose action $\alpha((g_1,g_2),v_1\oplus v_2)$ on $V_1\oplus V_2$ is defined component-wise:
        \begin{itemize}
            \item $\alpha((g_1,g_2),v_1\oplus v_2)=\phi_1(g_1,v_1)\oplus\phi_2(g_2,v_2)$, $\forall$ $g_1 \in G_1$, $g_2 \in G_2$, $v_1\in V_1$, $v_2\in V_2$, $\phi_1:G_1\times V_1\xrightarrow{}V_1$ and $\phi_2:G_2\times V_2\xrightarrow{}V_2$ .
        \end{itemize}
    \end{definition}
    Particularly, for matrix representations $(V_1,\rho_1)$ and $(V_2,\rho_2)$ of a group $G$, $\rho_1\oplus\rho_2:G\times G\xrightarrow{}\mathbf{GL}(V_1\oplus V_2)$ is given by $(g_1,g_2)\mapsto
    \begin{pmatrix}
    \rho_1(g_1) & \mathbf{O} \\ 
    \mathbf{O} & \rho_2(g_2)
    \end{pmatrix}$.
\end{frame}

\begin{frame}{Direct Product of representations}
    \begin{definition}[1]
        \boldtext{Direct product} of representations $\rho_1$:$G_1\xrightarrow{}\mathbf{GL}(V_1)$ and $\rho_2$:$G_2\xrightarrow{}\mathbf{GL}(V_2)$ is a group homomorphism $\rho_1\otimes\rho_2:G_1\times G_2\xrightarrow{}\mathbf{GL}(V_1\otimes V_2)$ such that $(\rho_1\otimes\rho_2)((g_1,g_2),v_1\otimes v_2)=\rho_1(g_1,v_1)\otimes\rho_2(g_2,v_2)$, $\forall$ $g_1 \in G_1, g_2 \in G_2$, $v_1\in V_1, v_2\in V_2$.
    \end{definition}
    \begin{definition}[2]
        \boldtext{Direct product} of representations $(V_1,\rho_1)$ and $(V_2,\rho_2)$ of groups $G_1$ and $G_2$, respectively, is a linear representation $(V_1\otimes V_2,\rho)$ of the group $G_1$$\times$$G_2$ whose action $\alpha((g_1,g_2),v_1\otimes v_2)$ on $V_1\otimes V_2$ is defined component-wise:
        \begin{itemize}
            \item $\alpha((g_1,g_2),v_1\otimes v_2)=\phi_1(g_1,v_1)\otimes\phi_2(g_2,v_2)$, $\forall$ $g_1 \in G_1$, $g_2 \in G_2$, $v_1\in V_1$, $v_2\in V_2$, $\phi_1:G_1\times V_1\xrightarrow{}V_1$ and $\phi_2:G_2\times V_2\xrightarrow{}V_2$ .
        \end{itemize}
    \end{definition}
    For $G_1=G_2$, $\rho_1\otimes\rho_2$ is also called a \italictext{tensor product} of the representations.
\end{frame}

\begin{frame}{Subrepresentation and Reducibility}
    \begin{definition}
        A \boldtext{subrepresentation} $\sigma:G\xrightarrow{}\mathbf{GL}(W)$ of a representation $\rho:G\xrightarrow{}\mathbf{GL}(V)$ is a map that is preserved by group action, i.e. $\forall$ $w\in W$ and $g\in G$, $\sigma(g,w)\in W$, where $W$ is a linear subspace of $V$.
    \end{definition}
    Clearly $\sigma$ is also a representation of the group $G$.\\
    Indicatively, the subspace $W$ itself is referred to as the subrepresentation.\\
    A trivial (sub)representation $V^G(g):V\xrightarrow{}V$ is the map $V^G(g):v\mapsto v$.
    \begin{definition}
        A non-trivial representation $(V,\rho)$ is called an \boldtext{irreducible representation} (or \italictext{irrep}) if it has exactly two subrepresentations, viz. $(\{\mathbf{0}\},\sigma)$ and $(V,\rho)$.
    \end{definition}
    \begin{definition}
        A \boldtext{reducible representation} is a non-trivial representation with more than two subrepresentations.
    \end{definition}
\end{frame}

\begin{frame}{Subrepresentation and Reducibility (contd.)}
    \begin{theorem}
        Given two linear representations $(V,\rho_V)$ and $(W,\rho_W)$ of a group $G$ and a homomorphism $\alpha:V\xrightarrow{}W$, such that $\forall$ $g\in G$, $\alpha\circ\rho_V(g)=\rho_W(g)\circ\alpha$, $\mathbf{ker}(\alpha)$ is a subrepresentation of $V$ and $\mathbf{im}(\alpha)$ is a subrepresentation of $W$. 
    \end{theorem}
    \begin{definition}
        A representation $V$ is said to be \boldtext{decomposable} if it is a direct sum of its subrepresentations $(\neq\{0\},V)$. Else, it is said to be \boldtext{indecomposable}.
    \end{definition}
    In terms of matrix representations, a representation is \italictext{decomposable} if all representative matrices can be block diagonalized upon a similarity transformation by the same invertible matrix, else it is \italictext{indecomposable}.\\
    Each block is a subrepresentation of the matrix representation.
    \begin{definition}
        Matrix representations which are associated with each other through a similarity transformation are called \boldtext{equivalent representations}.
    \end{definition}
\end{frame}

\begin{frame}{More on irreducible representations}
    \begin{theorem}
        Irreducible representations are always indecomposable.
    \end{theorem}
    The converse holds for finite-dimensional complex representations of finite groups, but may not be true in general.\\
    In terms of matrix representations, an irrep is one for which there exists no unitary transformation (similarity transformation through a unitary matrix) that will block diagonalize all the representative matrices simultaneously.\\
    \begin{theorem}
        The tensor product of irreps $(V_1,\rho_1)$ and $(V_2,\rho_2)$ of a group $G$ is not an irrep of $G$, in general. Although it is an irrep of the group $G\times G$.
    \end{theorem}
    $\rho_1\otimes\rho_2$ is decomposed as a direct sum of irreducible representations of $G$ using the \italictext{Clebsch-Gordan procedure}.
\end{frame}

\begin{frame}{More on irreducible representations (contd.)}
    \begin{theorem}
        Direct product of irreps $(V_1,\rho_1)$ and $(V_2,\rho_2)$ of distinct groups $G_1$ and $G_2$, respectively, is an irrep $(V_1\otimes V_2,\rho)$ of the group $G_1\times G_2$. In fact, all irreps of $G_1\times G_2$ arise as direct product of irreps of $G_1$ and $G_2$, respectively.
    \end{theorem}
    \begin{theorem}
        If $(V,\rho_V)$ is a representation of a finite group with a subrepresentation $(W,\rho_W)$, there exists a complementary subrepresentation $(W^\prime,\rho_{W^\prime})$ of $(V,\rho_V)$ such that $\rho_V=\rho_W\oplus\rho_{W^\prime}$.
    \end{theorem}
    \begin{corollary}[Maschke's theorem]
        Every representation of a finite group is a direct sum of its irreps.
    \end{corollary}
    Such representations are called \italictext{completely reducible representations}.\\
    Every representation of a finite group is equivalent (through a similarity transformation) to a unitary representation, which is completely reducible.
\end{frame}

\begin{frame}{Dual Representation}
    \begin{definition}
        The \boldtext{dual representation} $\rho^*:G\xrightarrow{} \mathbf{GL}(V^*)$ over the dual vector space $V^*$ is defined with respect to the representation $\rho:G\xrightarrow{} \mathbf{GL}(V)$ over vector space $V$ such that $\rho^*$ maps an element of $G$ to the transpose of the image of the inverse of that element under $\rho$, i.e. $\rho^*:g\mapsto{\rho(g^{-1})}^\intercal$.
    \end{definition}
    The dual representation is also known as the \italictext{contragredient representation}.
    \begin{proposition}
        The dual representation $\rho^*$ of a group $G$ over the dual vector space $V^*$ is irreducible iff the associated representation $\rho$ of $G$ over the associated vector space $V$ is irreducible.
    \end{proposition}
    \begin{proposition}
        The dual ${\rho}^{**}$ to the dual $\rho^*$ to any representation $\rho$ is isomorphic to $\rho$.
    \end{proposition}
    Although, $\rho^*$ need not be isomorphic to $\rho$.
\end{frame}

\begin{frame}{Exercise 5}
    \begin{enumerate}
        \item Argue in favour or opposition:
        \begin{itemize}
            \item For a finite order matrix transformation $T:V\xrightarrow{}V$ of a finite dimensional vector space $V$, $T$ is not necessarily diagonalizable. \textbf{(F)}
            \item Commutator $[A,B]$ of finite matrices $A$ and $B$ is traceless. \textbf{(T)}
        \end{itemize}
        \item Prove the following:
        \begin{itemize}
            \item If $(V,\rho)$ is a matrix representation of a finite group $G$ over a finite dimensional vector space $V$, $\rho(g)$ is diagonalizable, $\forall$ $g\in G$.
            \item For a unitary representation $\rho$ of a group $G$, the representative $\rho^*(g)$ in the dual representation $\rho^*$ is the complex conjugate of $\rho(g)$, $\forall$ $g\in G$. 
        \end{itemize}
    \end{enumerate}
\end{frame}

\section{Unit VI - Character Theory of Groups}

\begin{frame}{Schur's Lemma}
    \begin{lemma}
        For two irreducible representations $(V,\rho_V)$ and $(W,\rho_W)$ of a group $G$, the homomorphism $\alpha:V\xrightarrow{}W$, such that $\forall$ $g\in G$, $\alpha\circ\rho_V(g)=\rho_W(g)\circ\alpha$, is either identically zero or an isomorphism (invertible).
    \end{lemma}
    Clearly, if the two representations are not isomorphic, for the map $\alpha$ to remain equivariant, it must be identically zero.
    \begin{corollary}
        If $(V,\rho_V)$ and $(W,\rho_W)$ are isomorphic, the space of all homomorphisms $\alpha:V\xrightarrow{}W$ is one-dimensional.
    \end{corollary}
    \begin{corollary}
        For $V=W$ and $\rho_V=\rho_W$, any endomorphism $\alpha:V\xrightarrow{}V$ is a scalar multiple of the identity transformation in $V$ $(\mathbf{id}_V)$, i.e. $\alpha=\lambda\mathbf{id}_V$, $\lambda\in\mathbb{C}$.
    \end{corollary}
\end{frame}

\begin{frame}{Schur's Lemma (contd.)}
    \begin{corollary}
        If a finite group $G$ has a faithful irreducible representation, then its center $Z(G)$ is a cyclic subgroup of $G$, isomorphic to that of $(\mathbb{C}\backslash \{0\},\times)$.
    \end{corollary}
    \begin{corollary}
        All irreducible representations of finite abelian groups are one-dimensional.
    \end{corollary}
    \begin{corollary}
        A finite abelian group $G$ has exactly $\lvert G\rvert$ irreducible representations.
    \end{corollary}
\end{frame}

\begin{frame}{Character}
    \begin{definition}
        The \boldtext{character} of a linear representation $\rho:G\xrightarrow{}\mathbf{GL}(V)$ (where $V$ is a vector space over $F=\mathbb{R}, \mathbb{C},$ etc.) is the function $\chi_\rho:G\xrightarrow{}F$ given by $\chi_\rho(g)=\mathbf{Tr}(\rho(g))$, $\forall g\in G$.
    \end{definition}
    For a matrix representation of a group, the character of the representation for any given element is the trace of the representative matrix for that element in that representation.\\
    A character $\chi_\rho$ is called \italictext{irreducible} if $\rho$ is an irreducible representation.\\
    \begin{definition}
        \boldtext{Kernel} of a character is defined as $\mathbf{ker}(\chi_\rho)\coloneqq\{g\in G\mid \chi_\rho(g)=\chi_\rho(e)\}$, where $e$ is the identity in $G$.
    \end{definition}
    Clearly, $\mathbf{ker}(\chi_\rho)=\mathbf{ker}(\rho)$, hence, a normal subgroup of $G$.\\
    A character $\chi_\rho$ is \italictext{faithful} if $\mathbf{ker}(\chi_\rho)$ is trivial.
\end{frame}

\begin{frame}{Character (contd.)}
    \begin{theorem}
        Character of a direct sum (product) representation of representations $\rho$ and $\sigma$ is the sum (product) of the characters $\chi_\rho$ and $\chi_\sigma$ of the respective representations, i.e. $\chi_{\rho\oplus\sigma}=\chi_\rho+\chi_\sigma$ and $\chi_{\rho\otimes\sigma}=\chi_\rho\cdot\chi_\sigma$.
    \end{theorem}
    \begin{theorem}
        For a representation $\rho$ of a group $G$, $\chi_{\rho^*}(g)=\chi_{\rho}(g^{-1})=\overline{\chi_{\rho}(g)}$, $\forall$ $g\in G$.
    \end{theorem}
    Dimension of a representation $\rho$ is called the \italictext{degree} of the character $\chi_\rho$.
    \begin{theorem}
         The number of complex irreducible characters of a group $G$ is equal to the number of conjugacy classes of $G$, and their degrees divide $\lvert G\rvert$, $\lvert G:Z(G)\rvert$.
    \end{theorem}
\end{frame}

\begin{frame}{Class Function}
    \begin{definition}
        \boldtext{Class functions}
    \end{definition}
    \begin{definition}
        \boldtext{Inner product}
    \end{definition}
\end{frame}

\begin{frame}{Schur Orthogonality Relations}
    Let $\Gamma^{(\alpha)}$ be the irreducible matrix representations (of finite dimensions ${l_\alpha}$) of a group $G$, indexed by some $\alpha,\beta,\dotsc\in\mathbb{N}$, which are unitary, i.e. ${\Gamma^{(\alpha)}(g)}^{\dagger}\Gamma^{(\alpha)}(g)=\mathbb{I}\text{ ,  or , }\sum\limits_{k=1}^{l_\alpha}{\Gamma^{(\alpha)}(g)}_{km}^*\Gamma^{(\alpha)}(g)_{kn}=\delta_{mn},\text{ }\forall\text{ }g\in G$ .
    \begin{theorem}
        Matrix elements of finite unitary irreps satisfy the orthogonality relations,
        \vspace{-0.75em}
        \begin{align*}
            \sum\limits_{g\in G}\;{\Gamma^{(\alpha)}(g)}_{mn}^*\;\Gamma^{(\beta)}(g)_{m^\prime n^\prime}\;=\;\frac{\lvert G\rvert}{l_\alpha}\; \delta_{\alpha\beta}\;\delta_{mm^\prime}\;\delta_{nn^\prime}\;\;. 
        \end{align*}
    \end{theorem}
    Due to the vast applicability of the relations, this is also called the \italictext{Great Orthogonality Theorem}.
\end{frame}

\begin{frame}{Character Table}
    \begin{definition}
        
    \end{definition}
\end{frame}

\begin{frame}{Exercise 6}
    \begin{enumerate}
        \item Argue in favour or opposition:
        \begin{itemize}
            \item Character of a representation of a group is a group homomorphism. \textbf{(F)}
            \item Character of an irrep of an abelian group is a homomorphism. \textbf{(T)}
        \end{itemize}
        \item Prove the following:
        \begin{itemize}
            \item The character of a representation of a group $G$ maps the identity in $G$ to the dimension of that representation.
            \item Character of a given representation is same for every member of a class.
            \item Isomorphic representations have the same character.
        \end{itemize}
    \end{enumerate}
\end{frame}

\end{document}
